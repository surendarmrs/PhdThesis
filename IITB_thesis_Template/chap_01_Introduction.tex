\chapter{Introduction}
\section{Background}

\label{sec:background}

\lettrine[findent=2pt]{\textbf{S}}{}now is the backbone of the hilly, mountainous areas of the world because of its freshwater storage. It supply water in lakes, rivers and large portions of the world's highly populated zones like India. Snow is primarily an ice particle formed by sublimation of vapor in the atmosphere or a collection of loosely bonded ice crystals deposited from the atmosphere. It originates in clouds when temperatures are below the freezing point and water vapor in the atmosphere condenses directly into ice without going through the liquid stage. Once an ice crystal has formed, it absorbs and freezes additional water vapor from the surrounding air, growing into a snow crystal or snow pellet, which then falls on the Earth. Snowfall requires two specific weather conditions: low temperatures and moisture in the atmosphere. It is most common in high altitudes and high latitudes, particularly in the mountainous regions of the Northern and Southern Hemispheres. Snow cover over the Indian Himalaya is shown in Figure~\ref{fig:snow image}.
\begin{figure}[!htbp]
	\centering
	\subfloat[]{\includegraphics[width=0.48\textwidth]{Figure_General/snow1.JPG}} \hspace{1mm}
	\subfloat[]{\includegraphics[width=0.48\textwidth]{Figure_General/snow2.JPG}} \\
	\subfloat[]{\includegraphics[width=0.48\textwidth]{Figure_General/snow4.JPG}} \hspace{1mm}
	\subfloat[]{\includegraphics[width=0.48\textwidth]{Figure_General/snow3.JPG}}
	\caption [Snow cover regions over the Indian Himalaya] {Snow cover in the Indian Himalaya.} 
	\label{fig:snow image}
\end{figure}

\begin{figure}[!htbp]
	\centering
	\subfloat[]{\includegraphics[width=0.48\textwidth]{Figure_General/snow_crystal_1.JPG}} \hspace{1mm}
	\subfloat[]{\includegraphics[width=0.48\textwidth]{Figure_General/snow_crystal_2.JPG}} \\
	\subfloat[]{\includegraphics[width=0.48\textwidth]{Figure_General/snow_crystal_3.JPG}} \hspace{1mm}
	\subfloat[]{\includegraphics[width=0.48\textwidth]{Figure_General/snow_crystal_4.JPG}}
	\caption [Different types of snow flakes] {Different types of snow flakes \small(accessed from SnowCrystals.com as on 19 Jan. 2016).} 
	%\caption*{\small(source :SnowCrystals.com)}
	\label{fig:snow crystals}
\end{figure}

Changes in climate can affect the amount and duration of the snowfall in winter season. The change in snow cover area and the amount of snow that melts affect water supplies and hydro power generation. An intrinsic relationship exists between snow in cold region and the discharge in the emanating river system. The snow cover is a temporal intermediate storage, releasing the ecologically and economically important melt water in warmer periods. In hydrological investigations, modeling and forecasting of snow melts runoff require timely information about spatial variability of snow properties like snow cover, snow wetness, snow density, snow depth and grain size. The microscopic view of different snow flakes are shown in Figure~\ref{fig:snow crystals}. Several organizations and agencies will be benefited from the precise quantitative information over large area about snowpack parameters, including climate and ocean modeling agencies, hydropower industries, snow and avalanche research centers, local weather forecast agencies. Moreover, north Indian sub-continent livelihoods are the prime beneficiaries because Himalayan snow and glaciers are the primary river sources for this region. In addition to that, these parameters would help in the estimation of snow melt run-off, which will be useful for predicting floods in the Himalayan catchment areas. 

Over major portions of the middle and high latitudes, and at high elevations in the tropical  latitudes, snow and alpine glaciers are the largest contributors to runoff in rivers and to ground-water recharge. Snow and ice also play important interactive roles in regional climates, because the snow has a higher albedo than any other natural surface. The snow covered areas being mostly at higher altitude and remote places, the data collection by conventional and ground-based methods are difficult, both in terms of logistic and sporadic occurrences. Even when available, snow cover data represent only point measurements of field observations which may or may not be representative of a large area or basin. Due to the strong spatial and time dependent dynamics of snow cover, high temporal observations are necessary. 

From this point of view, satellite remote sensing can be used which has the ability to monitor the snow pack parameters over a large area. Remote sensing is the science of acquiring, processing and analyzing the information about the Earth's surface without actually being in contact with it. This is done by sensing the energy emitted or reflected from the Earth's surface.  The microwave portion of the electromagnetic spectrum has a special importance in satellite remote sensing because of its all-weather, day and night observation capability. The microwave spectrum covers the range from approximately 1cm to 1m in wavelength. Longer wavelength microwave radiation can penetrate through cloud cover, haze, dust etc. Even during heavy rainfall the longer wavelengths of microwave signals are not susceptible to atmospheric scattering unlike shorter optical wavelengths. This property allows detection of microwave energy under almost all weather and environmental conditions allowing data acquisition at all time.

The principle of passive Microwave Remote Sensing (MRS) is similar to that of thermal remote sensing. All objects emit electromagnetic energy which they absorbs from the sun; but the energy from microwaves is generally very low in magnitude. Passive MRS systems measure this naturally emitted energy within its field of view. This emitted energy is dependent on the temperature and dielectric properties of the objects. Passive microwave sensors are typically radiometers. The energy available from them is quite small compared to optical wavelengths because of their higher wavelengths.Thus, the field of view must be large to detect enough energy to record a signal. Most passive microwave sensors (radiometers) are therefore characterized by low spatial resolution.

Moreover, information from radar images is quite different from that received by radiometer. Due to these differences, the radar offers different perspectives of the Earth’s surface. The radar backscatter coefficient has been shown to be very useful for quantitative estimation of snow pack parameters. Ground-based experiments were conducted to study the effect of dry and wet snow on backscattering coefficient of the terrain~\citep{Stiles1980}. In general, the backscatter received by the SAR antenna from the snowpack can be modeled as a combination of: (1). the surface scattering at the air-snow interface, (2). the underlying ground surface scattering, (3). the volume scattering within the snowpack and (4). the snow-soil interface scattering. The measured backscattering coefficient is a function of several snow and soil parameters: snow layer, thickness, snow temperature, snow wetness, snow density, surface roughness (air-snow interface as well as snow-ground interface)~\citep{ulaby1986microwave}. Radar backscattering effects on geographical areas, relief, aspect angle, layover and shadow have also been studied in~\citep{Koskinen97,Nagler2000}. The snow volume scattering has been modeled by a discrete particle model which was experimentally justified at C-band~\citep{Kendra98,Koskinen2000}. A semi-empirical model for radar backscattering coefficient of snow covered ground was developed at 35~GHz and 95~GHz frequencies~\citep{Ulaby95}. This model relates the backscatter coefficient to the incidence angle and the snowpack parameters (snow depth, crystal size and liquid water content) for each linear polarization.  

The electromagnetic (EM) response of any material is defined in terms of its magnetic permeability ($\mu$) and its relative complex dielectric constant ($\varepsilon$)~\citep{von1954dielectric}. The $\mu$ of snow is equal to that of the free space, $\mu=\mu_0$ and therefore, the propagation of EM wave in snow is only a function of $\varepsilon=\varepsilon^{'}-j\varepsilon^{''}$, ($j=\sqrt{-1}$). In order to characterize snow, which is a heterogeneous mixture of air, ice and liquid water, we need to understand the dielectric properties of the mixture. The dielectric behavior of water and ice is described by a Debye type relaxation spectrum~\citep{stiles1980dielectric}. Snow is classified as wet or dry depending upon the amount of liquid water content. Dry snow consists of ice particles and air, whereas wet snow contains liquid water as a third component. Microwaves strongly respond to this change in liquid water content in snow~\citep{hallikainen1986dielectric,tiuri1984complex,Hallikainen87}. The estimation of snowpack parameters requires a good understanding of the scattering mechanisms from the snowpack. Scattering from the air-snow surface and the uppermost layers are effective for wet snow estimation. The attenuation of a propagating EM wave is given in terms of the volume extinction coefficient ($\kappa_e$) and the penetration depth is defined as, $\delta_{p}=1/\kappa_e$. 

For dry snowpack, the backscatter contribution from the air-snow surface is small and thus can be neglected. The total backscatter contribution is a combination of snow volume and snow-ground surface. For dry snow, the penetration depth, $\delta_p\approx$10~m at 10~GHz and decreases to 1~m at 40~GHz~\citep{Rott85}. Volume scattering in snow is due to dielectric discontinuities. Volume scattering from thin, dry snow cover is undetectable at wavelengths longer than 10 or 15 cm, however, the level of scattering is dependent on the amount of snow on the ground (more snow implies more dielectric discontinuities for scattering )~\citep{bernier1987microwave}. The extinction coefficient, $\kappa_e$ which is inversely related to $\delta_p$ is equal to the sum of the absorption coefficient ($\kappa_{a}$) and scattering coefficient ($\kappa_{s}$). For wet snow, $\kappa_{e}\approx\kappa_{a}$ at microwave frequencies, as the absorption losses are much larger than the scattering losses. Due to this, $\delta_p$ is of the order of 1 or 2 wavelengths and hence, the snow-ground scattering may be neglected. For C-band SAR, the backscattering signal from wet snow is dominated by the scattering from air-snow interface and snow volume medium~\citep{Shi95wetness}. Several models have been developed to predict backscattering from rough surfaces: (1). the physical optics model, (2). the geometric optics model, (3). the small perturbation models (SPMs) and (4) the integral equation model (IEM) which is valid for a wide range of surfaces~\citep{Fung92,fung1994microwave}. Several studies have shown both positive and negative correlations between the backscatter coefficient and snow wetness possibly due to surface roughness~\citep{stiles1980dielectric,shi1992radar}. Several algorithms for snow covered area retrieval have shown a negative correlation between the backscatter coefficient and wetness values for reasonable snow wetness and depth~\citep{Koskinen97,Nagler2000,Guneriussen2001b}. Even though, there are many studies that have shown an empirical relationship between the radar backscattering coefficients and snowpack parameters, no standard method is available to study the quantitative estimation of snowpack parameters over the high topographic regions like Indian Himalaya till date.   

Even though, few studies~\citep{Shi95wetness,shi2000depth,Shi2000} have used full polarimetry data for snowpack parameter estimation, but the complete polarization information were not utilized. However, only the first order backscattering coefficients to invert snow parameters were considered. Moreover, SAR measurements can be efficiently used to infer bulk properties of snowpack. In comparison to conventional single-channel SAR, the introduction of SAR polarimetry can significantly improve the quality of the results. These improvements are due to the quantitative ability to utilize full vector nature of the Em wave for scattering mechanisms. Hence, remote sensing using polarimetric SAR data has great potential in determining the extent and the properties of snowpack. There have been many snow and glacier studies over the Indian Himalayan region using fully polarimetric SAR data~\citep{singh2011a,singh2012a,singh2012b,singh2014a,singh2014b}. Apart from these studies, dual-~polarimetric (HH/VV) coherent TerraSAR-X (X-band) and full polarimetric Radarsat-2 (C-band)  data are used for the development of inversion techniques for the estimation of snow parameters in this thesis. To consider the limitations and research gaps in the field of snow parameters estimation from SAR data, the following important points have been the motivation to choosing the research objectives.

\section{Motivation}
\begin{itemize}
	\item SAR polarimetry is a very active research area in radar remote sensing in which there is an increased need to explore the potential for quantitative estimation of bio-/geo-physical parameters.
	
	\item Many works have reported the study of snow parameters using SAR data, among which very few studies barely attempted to retrieve any quantitative (for example snow wetness, snow density, snow grain size etc.) information.
	
	\item There is no proven precise methodology for the estimation of snowpack parameters using full polarimetric SAR data over the Indian Himalayan region. 
	
	\item The data obtained from the new generation advanced full-polarimetric SAR sensors along with the advanced polarimetric decomposition techniques, provide an opportunity to develop improved algorithms for snow pack parameters estimation. 
	
\end{itemize}
\section{Research objectives}
In this thesis, polarimetric SAR data is used for the estimation of snowpack parameters over the Indian Himalayan region. In this context algorithms are developed for the estimation of snow wetness, snow surface dielectric constant and density. The proposed objectives are as follows, 

\begin{description}
	\item[$\ 1 (a).$ ] Estimation of snow wetness from dual polarimetric $(\mbox{HH}+\mbox{VV})$ coherent X-band data.
	\item[$\ 1 (b).$ ] Estimation of snow wetness from full polarimetric $(\mbox{HH}+\mbox{HV}+\mbox{VH}+\mbox{VV})$ SAR data.
	\item[$\ 1 (c).$ ] Estimation of snow surface dielectric constant from full polarimetric $(\mbox{HH}+\mbox{HV}+\mbox{VH}+\mbox{VV})$ SAR data.
	\item[$\ 2.   $ ] Development of an algorithm for estimation of snow density from full polarimetric $(\mbox{HH}+\mbox{HV}+\mbox{VH}+\mbox{VV})$ C-band SAR data.
\end{description}

%\begin{enumerate}
%	
%\item Development of an algorithm for 
%	\begin{enumerate}
%	  \item Estimation of snow wetness from dual polarimetric $(\mbox{HH}+\mbox{VV})$ coherent TerraSAR-X X-band (9.65 GHz) data.
%	  \item Estimation of effective snowpack wetness from full polarimetric $(\mbox{HH}+\mbox{HV}+\mbox{VH}+\mbox{VV})$ Radarsat-2 (5.4 GHz) data.
%	  \item Estimation of snow surface dielectric constant from full polarimetric $(\mbox{HH}+\mbox{HV}+\mbox{VH}+\mbox{VV})$ SAR data.
%	\end{enumerate} 
%\item Development of an algorithm for estimation of snow density from full polarimetric $(\mbox{HH}+\mbox{HV}+\mbox{VH}+\mbox{VV})$ Radarsat-2 (5.4 GHz) data.
%\end{enumerate} 

These objectives are accomplished over the Indian Himalayan region for which the data is acquired and measurements area recorded from the field campaigns conducted during January to March 2012-2014. In the Indian Himalayan region the snowfall normally occurs during December to March from an altitude of 2000 m above the mean sea level. The expected snow wetness during Jan.-- Feb. is around 2--6 $\%$ by volume because of fresh snowfall and average minimum temperature. The snow density variation mainly depends on the temperature which produces the snow melt-freeze cycle. In the Indian Himalayan region, the mean minimum temperature in the month of January is around -15$^\circ$C-0$^\circ$C and the mean maximum temperature in the month of June is around 20$^\circ$C-30$^\circ$C. The mean high and low temperatures in the month of February over the study area are around 11.7$^\circ$C  and -0.7$^\circ$C respectively.
	  
\section{Thesis outline}
	The subject matter of the thesis is presented in the following five chapters, 
\begin{enumerate}[label=\checkmark]
\item	Chapter-1 gives an overview of the advantage of polarimetric SAR systems for snowpack parameters estimation. It also describes an outline of the snowpack parameters and their characteristics with respect to the electromagnetic waves, and also emphasizes the motivation of this research and objectives.
\item	Chapter-2 elucidates the principle and important parameters of SAR system and the characteristics of snowpack parameters. A thorough investigation of snowpack characterization studies, polSAR decomposition techniques and their advancements are included in this chapter. 
\item	Chapter 3 describes all the new developments of methodologies for the estimation of snow parameters from available SAR systems in separate subsections. All the new developments are presented with detailed flowcharts and derivations. The detailed description about the study area, in-situ field data collection and the data sets used for this study are incorporated in this chapter. 
\item Chapter-4 discusses the results obtained from all the algorithms and are presented in separate subsections along with detailed investigations using topographic and observatory measurements. Multi-temporal analyses of the results is also included in this chapter. 
\item	Chapter-5 highlights the new findings obtained by the utilization of SAR polarimetric data and conclusions arising out of this complete study are elucidated. The scope for future and continuation of this research work are also reported.  
%a new algorithm for the estimation of another important snowpack parameter, snow density is proposed using SAR polarimetry data. The detailed methodology with flow chart, snow density maps, thorough analysis of multi temporal variation of snow density within a season and the validation of the results are included in this section. 
%\item In Chapter-6, the proposed methodologies and the discussions of the results, including the important findings of the studies are summarized. The future scopes of the research works are proposed successively, following the conclusion on the basis of important extracts and understanding of the subject of interest.
\end{enumerate}

