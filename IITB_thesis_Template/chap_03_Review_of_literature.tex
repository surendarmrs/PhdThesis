\chapter{Review of Literature}
\section{Introduction}
In this Chapter, the emphasis of the discussion is on spaceborne active microwave remote sensing for snow parameters determination. A concise review of the physical basis behind SAR for snow parameters is presented along with a comprehensive exploration of all the major retrieval approaches and SAR polarimetry techniques as detailed in the literature to date. The next section elucidates the developments in Microwave Remote Sensing (MRS) imaging system particularly space borne Synthetic Aperture RADAR (SAR) systems. 
\section{Microwave Remote Sensing}
Microwave remote sensing is mainly classified as passive and active based on its imaging source. Passive systems are using naturally emitted, reflected or scattered microwave radiation from the earth surface but active systems are using its own illuminating source as shown in Figure~\ref{fig:microwave_img_sys}.

Objects at the earth surfaces emitting low energy level of microwave signals compared to the infrareds radiation.  

Objects at the earth's surface emit not only infrared radiation; they also emit microwaves at relatively low energy levels. When a sensor detects microwave radiation naturally emitted by the earth, that radiation is called passive microwave. Clouds do not emit much microwave radiation, compared to sea ice. Thus, microwaves can penetrate clouds and be used to detect sea ice during the day and night, regardless of cloud cover.

Microwave emission is not as strongly tied to the temperature of an object, compared to infrared. Instead, the object's physical properties, such as atomic composition and crystalline structure, determine the amount of microwave radiation it emits. The crystalline structure of ice typically emits more microwave energy than the liquid water in the ocean. Thus, sensors that detect passive microwave radiation can easily distinguish sea ice from ocean.

A major drawback to measuring passive microwave radiation is that the energy level is quite low. As a result, the radiation must be collected over a larger region. Details of the sea ice, such as leads, are not easily detected.

Because of their ability to detect sea ice through clouds during the day and night, passive microwave sensors provide nearly complete images of all sea ice-covered regions every day. These sensors have provided the most complete, long-term observations of sea ice, allowing scientists to detect notable changes in Arctic sea ice.

Sea ice observations from passive microwave sensors began in 1972 with the Electrically Scanning Microwave Radiometer (ESMR) aboard NOAA's Nimbus-5 satellite. In 1978, NASA's Scanning Multichannel Microwave Radiometer (SMMR) provided detailed, reliable information about sea ice. In 1987, a series of DMSP Special Sensor Microwave/Imager (SSM/I) sensors continued the time series, or long-term record, of sea ice data through present. In 2002, NASA launched the Aqua satellite, which carried the Advanced Microwave Scanning Radiometer-Earth Observing System (AMSR-E) sensor. The system's improved technology complemented the time series of sea ice data. ESMR, SMMR, SSM/I, and AMSR-E sea ice data are available from NSIDC.


 While passive systems make use of naturally emitted, reflected or scattered radiation from Earth’s surface, active systems are equipped with a transmitting unit and receive the signal backscattered or reflected from the illuminated terrain. An important class of active imaging sensors is radar system operating in the microwave region of the electromagnetic spectrum. The active operating mode makes these sensors independent from external illumination sources and additionally the fact of operating at the microwave region reduces drastically the impact of clouds, fog and rain on the resulting images. Thus, active radar imaging systems allow widely day and night all-weather imaging, an important requirement for continuos global scale monitoring (Lee and Pottier 2009). Synthetic Aperture RADAR (SAR) is an active system imaging the earth’s surface in day and night, in all weather, through cloud or haze and their spatial resolution is compatible with the topographic variations in the Himalayan region. SAR imagine systems provide a two-dimensional image of the radar reflectivity of a scene by illuminating it with microwave pulses and receiving the scattered field. There are two possible operation scenarios for such radar systems viz. monosatic and bi/multistatic. 
The monostatic radars use the same antenna for transmitting and receiving and therefore this configuration is known as monostatic configuration. It is the classical operation scenario for space and airborne radar systems up to now. SARs are mounted on a moving platform (such as an aeroplane, the space-shuttle or a satellite) and operated in a side-looking geometry (Figure 2.2). The SAR system is situated at a height h and moves with velocity VSAR. The antenna is aimed perpendicular to the flight direction, referred as azimuth direction. The antenna beam is directed slant-wise towards the ground with an incident angle and orthogonal to the moving direction, which is referred to as range or cross-track direction. The transmitted pulse is radiated by the antenna towards the ground along the antenna beam directions. The area covered by a single pulse (antenna beam) in ground range and azimuth direction is referred to as the antenna footprint. The backscattered signal is received by the receiver antenna and the receiver unit. The platform motion in the flight direction provides the scanning in the direction of the sensor trajectory, which is referred to as azimuth or along-track direction. The area scanned by antenna beam is the image swath (Lee and Pottier 2009).
\begin{figure}[!htbp]
	\centering
	\includegraphics[width=\columnwidth]{Figure_General/microwave_imaging_systems}
	\caption{Microwave Remote Sensing systems} 
	\caption*{\small(source :"Remote Sensing Illustration" by Arkarjun - Own work. Licensed under CC BY-SA 3.0 via Wikimedia Commons - \url{http://commons.wikimedia.org/wiki/File:Remote_Sensing_Illustration.jpg})}
	\label{fig:microwave_img_sys}
\end{figure}
\section{Introduction}

Radar polarimetry is the merging of the technological concept of Radar (radio detection and ranging) with the fundamental property of the full vector nature of polarized (vector) electromagnetic waves imaging.(gulab thesis)

Spaceborne sensors provide valuable information about the earth's surface and environment. Active microwave sensors are of particular interest for this task due to their high resolution and their ability to image through clouds and at night. The conventional spaceborne imaging radars implemented for long-duration missions (the SAR sensors ERS-1 , ERS-2, JERS- 1 , Radarsat) operate in a single-frequency single polarization mode. The advances in technologies in the last two decades have led to the development of imaging radar polarimetry, where the complete, complex scattering matrix for every resolution element is measured. This capability enables the measurement of a target's polarization properties, thus permitting a much more detailed understanding of the electromagnetic scattering process. Radar backscatter is strongly influenced by objects comparable in size to the radar wavelength. Therefore, a polarimetric sensor operating in various frequency bands provides information about the imaged target over a wide range of scales.


The field of synthetic aperture radar changed dramatically in the early 1980s with the introduction of advance radar techniques, such as polarimetry and interferometry. While both of these techniques had been demonstrated much earlier, radar polarimetry only became an operational research tool with the introduction of the NASA/JPL Airborne Synthetic Aperture Radar (AIRSAR) system in the early 1980s. Radar polarimetry was proven from space with the two Spaceborne Imaging Radar C-band and X-band (SIR-C/X) SAR flights on board the space shuttle Endeavour in April and October 1994. In this chapter, we describe the basic principles of SAR polarimetry and, thereby, provide tools necessary to understand SAR polarimetry applications, such as land classification.

\section{Factors Affecting the SAR Signal}
The backscattering coefficient ($\sigma$$^\circ$) is the fraction that describes the amount of average backscattered energy compared to the energy of the incident field. The intensity of $\sigma$$^\circ$ is a function of the physical and electrical properties of the target, along with the wavelength ($\lambda$), polarisation and incident angle ($\theta$) of the radar. 

It is understood that the backscattering signal from a seasonal natural snow cover is affected by three sets of parameters: (1) snow pack parameters including snow density, free liquid water content (snow wetness), particle size and size variation, characteristics of particle spatial distribution, and stratification; (2) subsurface parameters that include the dielectric and roughness properties at the snow - ground interface and (3) sensor parameters, which include the frequency, incidence geometry, and polarization;(~\cite{shi1992radar}) 

\subsection{Backscattering of Snow Pack Parameters}
The backscatter received by SAR antenna is the sum of surface scattering at air/snow interface, volume scattering within snow pack, scattering at snow/ground interface and volumetric scattering from underlying surface (if applicable). The surface scattering is directly proportional to polarization amplitude and volume scattering is directly proportional to transmissivity whereas the polarization amplitude and transmissivity depend on dielectric constant and local incident angle. The dielectric constant of snow is preliminarily a function of frequency, snow wetness, temperature and density (~\cite{ulaby1986microwave}, Hallikainen et al. 1986). In case of dry snow, volume scattering is governed by dielectric discontinuities which are created by the differences in electric properties of ice crystal and air. The volume scattering increases with snow grain size and inter layering and with an increase in the amount of snow. Age of snow may influence the SAR backscatter because older snow has larger grain size than new snow grains.
The situation is totally different in the case of wet snow (Ulaby and stiles 1980, Stiles and Ulaby 1980, Ulaby et al. 1986, Matzler and Schanda 1984). When at least top layer becomes wet (4$\%$-5$\%$ wetness by volume), the penetration capabilty of radar signal is reduced (Matzler and Schanda 1984). Matzler and Schanda (1984) reported that Radar scatterometer backscattering coefficient for like polarization (HH and VV) at look angle 400 from wet snow were 10 times lower than backscattering coefficient from dry snow and cross polarized backscattering coefficient were 100 times lower. The relationship between scattering mechanism and snow wetness has been discussed by Shi and dozier (1993).  Depending on which scattering component is dominated, radar measurements in response to snow wetness can be either positive or negative. The volume scattering is inversely correlated to snow wetness. The liquid water content mainly cause an increase of snow dielectric constant which results in a significant decrease in transmission at the air/snow interface and  high dielectric loss increases the absorption coefficient. The surface scattering is proportional to wetness (Shi and Dozier 1995b). For a given snow density range (100 to 550 kg/m3), the incident wavelength shift of radar signal in snowpack was described by Shi and Dozier (2000). Due to effect of snow density, the incident wavelength shift in snowpack was observed from 21 to 16 cm in comparison to 24 cm when it propagates in air. The complexity of relationship between the backscattering and snow parameters (i.e. wetness and density) makes it unrealistic to develop an empirical relationship between SAR backscattering and field measurements (Shi 2004, Singh et al. 2007). Hence there is need to develop inversion algorithm for snow pack parameters estimation for resolving this problem.   
\subsection{Effects of Sub-Surface (Underlying Snow) Parameters} 
\subsubsection{Dielectric Properties}
The electrical characteristics of terrain features interact with their geometric characteristics to determine the intensity of radar returns. One measure of an object’s electrical character is the complex dielectric constant, which is a parameter that indicates the reflectivity and conductivity of various materials. The complex dielectric constant describes the ability of materials to absorb, reflect and transmit microwave energy (Campbell 2002). Moisture content changes the electrical properties of a material, which in turn affects how the material will appear on a radar image. Bernier and Fortin (1998) obsereved the gradual thawing of the soil caused the soil moisture and the soil dielectric constant to increase  with a resulting observed increase of backscattering signal of the snow cover. In the microwave region of the spectrum, most natural materials have a dielectric constant in the range of 3 to 8 when dry, whereas water has a dielectric constant of approximately 80. This means that the presence of moisture in soil will result in significantly greater reflectivity (Ulaby et al. 1986).
\subsubsection{Surface Roughness}
The use of SAR data to retrieve surface roughness is of considerable importance in many domains, including agriculture, hydrology, and meteorology. Experimental results and studies using simulation models have shown that the radar signal is more sensitive to surface roughness at high incident angles than at low incident angles (Holah et al. 2005, Baghdadi et al.  2002, Fung and Cheng 1992, Ulaby et al. 1986). Gong et al. (1996) found that HH polarization is slightly more sensitive than VV polarization to soil surface roughness (Holah et al. 2005). Rough surfaces produce returns of relatively strong intensity for a wide range of depression angles (Li and Bryan 1983). 
\subsection{Effects of SAR System Parameters}  
Wavelength, polarization, incident angle, look direction and spatial resolution are some of the very important parameters of SAR instrument. Brief explanation of these parameters is presented here.
\subsubsection{Wavelength}
Wavelength is formally defined as the mean distance between two maximum (or minimum) successive peak of a pattern and is normally measured in micrometers ($\mu$m) or nanometers (nm) (Jensen 2004). Wavelength is inversely proportional to frequency, that is, longer wavelength has lower frequency while shorter wavelengths have higher frequency. When electromagnetic radiation passes from one substance to another, the speed of light and the wavelength change while the frequency remains the same. In remote sensing research, identifying the beginning and the ending wavelength and then attaching a description often specify a particular region of the electromagnetic spectrum. This wavelength interval in electromagnetic spectrum is commonly referred to as a band, channel, or region (Jensen 2004). Imaging radars normally operate within a small range of wavelengths with rather broad interval. The subdivisions of the active microwave region, as commonly defined are Ka, K, Ku, X, C, S, L, UHF, and P, in ascending order of wavelength. Radar wavelength has a fundamental influence on the interaction between the electromagnetic wave and the natural medium (Garestier et al. 2006). In principle, radar signals are capable of penetrating snow pack. In the absence of moisture, Penetration depth increases with the increase of wavelength (Singh and Venkataraman 2009). This means that longer wavelengths result in higher penetration (Campbell 2002). Lower frequency SAR data show promise for mapping only wet snow (Rott and Nagler 1993, Nagler 1996, Shi et al. 1994, Rao et al. 2006). This is because it is difficult to distinguish dry snow from bare ground using SAR data at the lower frequencies that are currently flown in space. X-band (2.4–3.75 cm, 8.0–12.5 GHz) and lower frequencies (longer wavelengths) are not generally useful for detecting and mapping dry snow because the size of snow particles is much smaller than the size of the wavelength. Thus, at these longer wavelengths, there is little chance for a microwave signal to be attenuated and scattered by the relatively small ice crystals comprising a snowpack (Waite and MacDonald 1970, Ulaby and Stiles 1980, 1981). In recent studies, it was found that the X-band image can discriminate fresh snow area from that of snow free or bare surfaces, but at C-band and L-band wavelength the signal returns may be quite similar, causing confusion in discriminating these categories. The preference of the X-band imagery with fixed channel over the C- band and L-band for general interpretation is a result of the greater sensitivity of the shorter wavelength to snow discrimination (Venkataraman et al. 2008). Wavelengths longer than 10–15 cm are not impeded as they move through most dry seasonal snowpacks (Bernier, 1987; Bernier and Fortin 1998). Volume scattering from a shallow, dry snow cover (SWE <20 cm) is undetectable at C-band (5.3 GHz, 5.6 cm), for example, because the backscatter is dominated by soil/snow scattering. Volume scattering in dry snow results from scattering at dielectric discontinuities created by the differences in electrical properties of ice crystals and air, and by ice lenses and layers. Atmospheric scattering is usually very small at intermediate and lower microwave frequency and can be neglected (Ulaby and Stiles 1980, Leconte et al. 1990, Leconte 1995). In the case of wet snow (Stiles and Ulaby 1980, Ulaby and Stiles 1980, Rott, 1984, Ulaby et al. 1986), when at least one layer of the snowpack (within the penetration depth of the radar signal) becomes wet (4–5 $\%$ liquid water content), the penetration depth of the radar signal is reduced to about 3–4 cm (or one wavelength at X-band) (Matzler and Schanda 1984). Thus, there may be high contrast between snow-free ground and ground covered with wet snow, thus making it possible to distinguish wet and dry snow when imaged with C-band SAR from space. Volume scattering increases with snow-grain size and internal layering, and with thickness of snow. Surface roughness is a relative concept dependent on incident microwave wavelength. As wavelength increases, surface roughness criteria will also change. In general, more surface features will appear smoother at longer wavelengths than at shorter wavelengths. Therefore, a SAR image will appear darker in longer wavelengths than in shorter wavelengths provided the other parameters are the same (Xia and Henderson 1997, Ban 2005). Theoretically, azimuth resolution is one half of the length of the radar antenna and range resolution equals to (pulse length × speed of light)/(2 × sine of the incidence angle) (Campbell 2002, Ban 2005). Therefore, wavelength will not affect the azimuth of range resolution of a SAR image (Ban 2005). Virtually, all earlier radar systems acquired images in a single band and a single polarization. Thus, relatively few studies have been carried out to examine the effect of wavelength on the detectability of snow. The wavelengths of the microwaves used in Radar are longer than those of visible light, and are less responsive to the boundaries between air and the droplets within the clouds. The result is that, for Radar, the clouds appear homogeneous with only slight distortions occurring when the waves enter and leave the clouds. Future research should be carried out on simultaneous use of multi-frequency radar imagery when available as combining different bands can provide more information than individual band (Ban 2005).
\subsubsection*{Incident Angle}
Angle of incidence is defined as the angle between the incident radar beam at the ground and the normal to the earth’s surface at the point of incidence (Lillesand et al. 2004). Depression angle is defined as the angle between an imaginary horizontal plane and a radar beam. Incident angle and depression angle are complementary angles, so their sum is 900 (Campbell 2002). Radar incident angle can influence the geographical feature identification and classification to a certain degree. For lower incident angle, the layover will distort the quality of SAR image in hilly terrain. For Higher incident angle, the radar layover is lower, but the shadow is too much, and it will lose some useful information (Li et al. 2005). Incident angle affects the detectability of target through its control of range resolution. On one hand, small incident angles produce poor range resolution than larger angles, as range resolution is inversely proportional to sine of the incident angle (Campbell, 2002). On the other hand, small incident angles provide higher spatial resolution in across track direction, improved imaging geometry in hilly and mountainous terrain (reduced foreshortening, less layover), improved thematic information content of the backscattering coefficient, and improved discrimination of small complex land cover surfaces (Wegmüller 2003). Experimental studies using simulation models have shown that the radar signal is more sensitive to surface roughness at high incident angles than at low incident angles (e.g. Baghdadi et al. 2002, Fung and Chan 1992, Ulaby et al. 1986, Holah et al. 2005). Ulaby and Batlivala (1976) found that for incident angles greater than 10º, the energy scattered back to the sensor increases with increasing surface roughness. Backscattering coefficient derived from ERS-1 data for glacier, firn and accumulation area decreases significantly at incident angle lower than 350 and remains almost constant between incident angle range 350 and 600. For area with vegetation and for rocks, backscattering coefficient decreases approximately linearly as incident angle increases (Nagler 1996). For Otztal site in Alps, using ERS-1 data with low incident angle (230) the layover and shadow area was found 36$\%$ and 42$\%$ in ascending and descending image respectively. However, the layover area in ASAR image with swath IS6 over part of Himalayan snow cover region was found 14$\%$. The layover and foreshortening areas are more for ENVISAT swath-2 mode data which is acquired at 230 incidence angle. For reducing these effects, radar data at higher incidence angles (ENVISAT swath-6 or 7) are to be acquired for snow studies (Singh et al. 2008). 

\subsubsection*{Look Direction}
Look direction, the direction at which the radar signal strikes the landscape, is important in both natural and manmade landscapes. In natural landscapes, look direction is especially important when terrain features display a preferential alignment. Look directions perpendicular to topographic alignment will tend to maximize radar shadow, whereas look directions parallel to topographic orientation will tend to minimize radar shadow. The extent of radar shadow depends not only upon local relief, but also upon orientations of features relative to flight path; those features positioned in the near-range portion (other factors being equal) will have the smallest shadows, whereas those at the far-range edge of the image will cast larger shadows (Campbell 2002). Therefore it is clear that there exists a close relationship between the look direction or radar azimuth and the orientation of the topographic feature. The same type of land-cover may appear very different on a radar image due solely to a different orientation relative to the radar look direction (Bryan 1979, Grey et al. 2003). 
\subsubsection*{Polarization}
Polarization is a property of certain types of waves that describes the orientation of their oscillations. The discovery of the phenomena of polarized electromagnetic energy dates back about AD 1000 when the Viking used crystals to observe the polarization of sky light under foggy condition light and were able to navigate in absence of sunlight. In 1669, the first known quantitative work on light observation was published by Erasmus Bartolinus. He was followed by C. Huygens who contributed most significantly to the field of optics by proposing the wave nature of the light and discovering polarized light (Lee and Pottier 2009). E.L Malus proved Newton’s conjecture that polarization is an intrinsic property of light (Konnen 1985). 
Light is a transverse electromagnetic wave. The electric and magnetic fields vibrate at right angles to the direction of propagation.  If all of the waves in a beam of light have their electric fields vibrating in the same direction, the light is described as polarized. That is, the polarization of a light wave describes the orientation of its electric field in space. Plane-polarized light has an electric field that oscillates in a specific plane perpendicular to the direction of propagation. Unlike randomly polarized light, the direction of the electric field vibration remains constant for plane-polarized light. There are two commonly mentioned special cases of polarization, horizontal polarization, where the electric field vibrates horizontally as the wave moves forward and vertical polarization, where the electric field vibrates vertically as the wave moves forward. Plane polarized light is also called linearly-polarized light because the electric field vector can be pictured as vibrating along a line in space (Figure 2.4). The direction of polarization can be described by the angle $/theta$ made by the electric field vector and the vertical axis. The electric field vector of circularly polarized light sweeps out a circle during each cycle of the wave (Figure 2.5). The magnitude of the electric field vector remains constant throughout each cycle, but its direction is continuously changing. The electric field can be thought of as spiraling around an axis as the wave moves forward, like the threads of an advancing screw. Elliptical polarization is the most general type of polarization. In fact, in the mathematical treatment of polarization, linear polarization and circular polarization are simply the two extremes of the elliptical case. As with circular polarization, the electric field vector rotates, but in this case the magnitude does not remain constant. That is, the tip of the electric field vector sweeps out an ellipse rather than a circle as the wave propagates. 
The polarization characteristics of electromagnetic energy recorded by a remote sensing system represent an important variable that can be used in many Earth resource investigations (Jensen 2004). It is possible to use polarizing filters on passive remote sensing systems (e.g., aerial cameras) to record polarized light at various angles. It is also possible to selectively transmit and receive polarized energy using active remote sensing systems such as RADAR (e.g., horizontal transmit, vertical receive- HV; vertical transmit, horizontal receive-VH; vertical transmit, vertical receive-VV; horizontal transmit, horizontal receive-HH).

Multiple-polarized RADAR imagery is an especially useful application of polarized energy (Jensen 2004). Shi and Dozier (1995a) found that the surface and volume interaction terms are the important scattering source for cross polarized signals. The surface-volume interaction terms under the independent assumption results in an over-estimation for HH polarization. The error increases as the surface roughness increases, which can be as large as by a factor of about 2 for a very rough surface. For VV polarization, however, it always over-estimates at small incident angle and under-estimates at large incident angle. The importance of multifrequency and multipolarization data for separating different snow and ice regimes on glaciers and for mapping accumulation and ablation areas has been analyzed by Rott et al. (1995). L- or C-band co- and cross-polarized channels in combination with X-band were found to be of main importance for snow and glacier applications. Multi-polarized, multi-frequency microwave data can suffice the needs up to some extent when a simplified form of Integral Equation Method (IEM) is used. The single band (C-band) SAR data is inadequate for estimating snow wetness taking more than one unknown parameter into consideration. The effect of the snow volume scattering albedo and surface roughness can be minimized to estimate the snow parameters using multipolarization backscattering coefficient (HH, VV and Re{HHVV}) measurement (Shi and Dozier, 1995,1996). Autret et al. (1989) and Chen et al. (1995) also reported that the influence of surface roughness can be minimised using the copolarised ratio (HH/VV). Hence multiple polarizations help to distinguish the physical structure of the scattering surfaces. Surface scatter dominates signal (Brag scattering) return of like-polarized (VV) imagery. Volume scatter (e.g. vegetation) is the major influence on signal return of cross-polarized (VH) imagery.
By comparing both the HH and HV images, the features and areas that represent regions on the landscape that tend to depolarize the signal can be identified. Such areas will reflect the incident horizontally polarized signal back to antenna as vertically polarized energy- that is they change the polarization of the incident microwave energy. Such areas can be identified as bright regions on the HV image and as dark or dark grey regions on the corresponding HH image. The polarization of the energy that would have contributed to the brightness of the HH image has been changed, so it creates instead a bright area on the HV image. The same information can be restated in a different way. A surface that is an ineffective depolarizer will tend to scatter energy in the same polarization in which it was transmitted; such areas will appear bright in the HH image and dark on the HV image (Campbell 2002). The alternating polarization mode of ASAR is capable of providing multi-polarimetric acquisitions by means of SAR acquisitions, where switching is made on polarizations instead of sub swaths. Each of these polarizing channels has varying sensitivities to different surface characteristics and properties. For example, the dynamic range of the like polarized component is larger than that of the cross-polarized component for snow cover area (Shi and Dozier 1997); this is in contrast to the measurement for forested areas, where the dynamic range of the cross-polarized component is larger than that of the like-polarized component (Dong et al. 1997).The information from alternative polarization SAR images can be of great help in the process of identification and classification of different types of scattering mechanisms, and where the penetration depth is different at different polarizations . However, use of satellite SAR polarimetry data sets for snow cover area analysis is only beginning and there are still many unknowns. Given the paucity of data analysis of snow covered area with multipolarized data sets and the variation in the earlier results much remains to be determined about the exact relationships between snow parameters and radar backscatter. Singh and Venkataraman (2007), in their study showed that the HH image showed a lower backscattering coefficient value than the VV images while studying the snow wetness in Himalayan snow covered region, India. 
\subsection*{Snow Pack Parameters}
Even the radar interaction with targets (e.g. snow) changes according to frequency (or wavelength), the radar interaction with targets also changes with radar polarisation. The interaction depends on the shape and orientation of the scatterers and the scattering mechanisms (i.e whether it is single or multiple scattering). The radar signature depends on the interaction and on whether the scatterers are principally of one type or a mixture of several types. Different polarization captures different information from the targets. Natural objects scatter an electromagnetic wave differently depending upon the incident wave polarization. By using imaging radar, the polarimetric characteristic can be measured for each pixel to understand the scattering mechanism associated with it. Even through the physical property of a pixel may not be easily recognized by the polarimetric response, the geophysical information associated with the scattering mechanism may be inferred from the polarimetric response. If single frequency monopolarization techniques are used, there is generally a considerable degree of ambiguity between different types of targets. To overcome this, the dimensionality of the observation needs to be increased. This can be achieved through the use of multiple frequencies and/or multiple polarizations. Multi-temporal and multiple incident angles are other ways of increasing the dimensionality of observation.  Snow cover mapping techniques based on single, multiple polarizations and multi-temporal single and multiple polarizations are given in sub section 2.4.1. Snow wetness and density studies using SAR are also reviewed in following sub sections
%\subsubsection*{Snow Cover Mapping}
%Ulaby et al. (1984) examined the effects of snow cover on the microwave backscattering in the 8 - 35 GHz region through the analysis of experimental data and by application of a semi-empirical model.  Through comparisons of measurement of backscattering data for different terrain surfaces with and without snow cover, they have evaluated the masking effects of snow as a function of snow water equivalence (SWE) and wetness.  The results indicate that with dry snow cover, it is not possible by a radar at 8.6 GHz, θ=50o to discriminate between different types of ground surface if SWE ≥ 20 cm.  However for the same incident angle and frequency, 3 cm SWE is sufficient to mask the underlying ground surface if the snow is wet (wetness ≥ 3 per cent).  As expected, the snow exhibits an even better masking effect at higher frequencies (Ulbay et al. 1984). 
%Rott et al. (1988) developed a snow mapping procedure in which actual image and simulated image are used. The simulated image is generated using local incident angle map of the area using DEM and some constant parameters which depends on surface type.  Rott et al. (1988) used C-band HH polarized data and simulated data with the following conditions to map snow covered area
%for      σ0hh/σ0sim             ≤ 1.8    snow and ice free
%for      1.8 <   σ0hh/σ0sim  < 3.5    glacier ice
%for       3.5 ≤ σ0hh/σ0sim                snow
%They compared SAR classified results with that of Landsat TM and found that snow surface was underestimated in the SAR classification.
%Since back-scattering signals from bare rock, soil and glacier are dominated by surface back-scattering, the back-scattering coefficients at a given incident angle mainly depend on dielectric properties and surface roughness. Back-scattering from wet snow covered area is dominated by both surface and volume scattering, depending on their physical parameters and surface roughness (Shi and Dozier 1993).  Generally, the dominant scattering mechanisms are surface back-scattering at small incident angle and volume back-scattering coefficients at large incident angle.  Since the surface of rock and soil is rougher than wet snow and glacier, the backscattering coefficients from rock or soil are greater. The roughness of glacier-ice surfaces is generally smaller than rock or soil but greater than wet snow, hence the magnitude of their back-scattering coefficient is between those from rock or soil and wet snow.  These differences in dielectric properties and surface roughness provide an opportunity to separate glacier, wet snow and bare rock or soil regions. (Shi and Dozier 1993).
%Airborne SAR data of L band, with HH polarization and C band with VV and HH polarization were analyzed by Shi and Dozier (1993) for snow mapping in the Otztal Alps, Austria.  The classification results obtained by them have been compared with Landsat TM data. The overall accuracies of classification from SARs were determined by the sum of the products of the percentage in TM scene and accuracy of SARs for each target.  Compared with TM data, the results at C band show a fairly good agreement for wet snow-covered area, 80 and 82$/%$ for VV and HH polarizations.  But they poorly identify glacier ice, with only 40$/%$ accuracy, because of difficulty in discriminating glacier from bare rock or soil when the glacier has either a rough surface or is partially covered by rock debris. In the ice-free regions, the classified SAR data accuracies reach 87 and 84$/%$.  The mis-classifications mainly occur in the areas where the rock is smooth.  The overall classification accuracies are 74 and 73$/%$ for VV and HH polarizations. The results from L-band HH polarization SAR data show the best discrimination of ice free regions with 90$/%$ accuracy, but at this frequency it is not usually possible to separate wet snow from glacier ice.  The overall classification accuracy is only 60$/%$.
%Airborne C-HH and simulated RADARSAT synthetic aperture radar imagery have been interpreted by Donald et al. (1993) to map wet snow cover in Ontario.  It has been found that Airborne SAR and simulated RADARSAT image provide 83$/%$ and 80$/%$ classification accuracy respectively. 
%When topographic information of the study area is available, SAR data can be radiometrically calibrated, thereby reducing the error caused by incorrectly estimating the illuminated area for obtaining radiometric properties of each target.  The remaining problem is the effect of local incident angle (Shi and Dozier 1993).  Generally the back scattering coefficient is considered as a product of
%σ0 (θi) = σ on  f(θi)				(2.1)	
%σ0n is the normalized back scattering coefficient related to the target back scattering properties and f(θi) represents the angular dependence of back scattering coefficients.  Hence the angular dependence can be reduced by normalizing the measured back scattering coefficients by the function f(θi).  The function f(θi) can be obtained from sampling the local incident angle map and SAR image (Shi and Dozier 1993, Shi et al. 1994). The snow classification is performed by applying threshold of σ0n. The classification based on σ0n at C-band VV gave an overall agreement of 66$/%$ when compared with Landsat TM classification. 
%Strozzi and Matzler (1998) carried out field experiments to measure backscattering coefficients of snow cover at Swiss and Austrian Alps with network analyzer based scatterometers at 5.3 (C band) and 35 GHz (Ka band).  These measurements were done in hh, vv, hv and vh polarization and for incident angle ranging from 0 to 70o.  At 40o incident angle, the combined use of active sensors at 5.3 and 35 GHz allows the discrimination of various snow cover situations, if multi temporal information is available.  
%Rott and Nagler (1992) and Rott and Nagler (1993) developed a method based on multi temporal ERS-1 SAR data. It is based on the fact that dry snow is not discriminated against snow free ground during winter as the dry snow is transparent at C-band. Images acquired during snow-free conditions are usually taken as reference image. However, wet snow strongly reduced the backscattered signal. Images acquired during wet conditions are usually known as wet snow images. The change in σ0 between snow and reference images leads to the detection of wet snow. By combining ascending and descending pass data, they could map snow cover even in layover and shadow areas. This snow mapping algorithm was further refined by Nagler (1995) and Nagler and Rott (2000) to improve the accuracy for steep topography. Nagler and Rott (2000) used threshold value -3dB for wet snow mapping after comparing the results with snow classification using Landsat TM data. The algorithm was also applied for snow mapping with ERS SAR in Northern Scotland and Northern Sweden (Caves et al. 1998) and in the mountainous regions of Norway (Malnes and Guneriussen 2002), and with slight modifications for agricultural areas in south-eastern Québec, Canada (Baghadi et al. 1997). Rao et al. (2006) also applied on Himalayan snow covered region for snow cover mapping. They found that the threshold -3dB has shifted to -2dB for Himalayan snow covered region and this algorithm shows the snow free area over glacier accumulation area.  
%A slightly different algorithm, also using multi-temporal SAR data, was suggested by Koskinen et al. (1997) which is optimized to detect snow in forested areas. By conducting a pixel wise comparison between the two reference images and the current image, the equation for the relative fraction of snow free ground is given by:(2.2)
%σ0i, σ0w, σ0g, are the backscattering during snow melt, and at the beginning of the snow melt period and for snow-free ground respectively.
%Luojus et al. (2006) demonstrated a two step technique for mapping wet snow cover over Boreal forest in Finland using ERS-2 multi-temporal data sets. The first step is the forest canopy compensation. This is done by nonlinearly fitting ERS-2 measured data with a semi-empirical forest-backscattering model and information with forest stem volume to estimate the backscattering signal at ground surface and the volume backscattering signal, the two-way transmissivity of the forest canopy. In the second step, the fraction of wet snow covered area SCA is calculated.
%
%(2.3)
%
%where σ0surf is the estimated backscattering coefficient at ground surface, σ0ground - ref is the reference signal from the snow-free ground, and σ0snow - ref  is the reference signal from the wet snow ground. One reference image describes the signal with full wet snow cover situation and the other describes the snow-free at the end of the snow melting season. In addition to these two reference images, this method requires forest-stem volume distribution.
%An Interferometry SAR technique for topographic mapping of surface does not only produce a high resolution DEM but also gives the information about changes on the surface during the repeat pass cycle of satellite from the correlation properties of the radar echo. The measurement of interferometer correlation provides the information of changes during the time scale of the satellite repeativity and on the order of SAR wavelength. The coherence measurement between two repeat passes provides another useful measurement for snow cover mapping over large area. Strozzi et al. (1999) analyzed the ERS-1/2 tandem data and observed that coherence can help separate wet snow from other surface where backscatters do not discriminate snow from other surface. 
%The basis of InSAR is phase comparison over many pixels. This means that the phase between scenes must be statistically similar. The coherence is a measure of the phase noise of the interferogram. The interferometric coherence is defined as the absolute value of the normalised complex cross correlation between the two signals. The correlation will always be a number between 0 and 1. If the pixels are similar this will result in a high correlation and more precise results. If the pixels are not similar i.e. not correlated at a certain degree then the phase will vary greatly and the result will be noise, we talk then about decorrelation.
%Theoretical and experimental studies have shown that coherence is determined by the spatial baseline of the interferometric pair, the time separation of the images, topographic effects, and noise sources. The change of backscattering characteristic at surface will cause the change of coherence degree through repeat pass. Integrating the backscattering intensity, we can make classification using the difference of coherence degree for different surface types.    
%The technique is based on coherence between two images acquired at different dates. The coherence may vary depending on the melting of snow, forest growth, precipitation, soil moisture and movement of the glaciers.  Through this technique, wet snow covered areas can be identified through coherence change.  Rao et al. (2008) observed poor coherence in most of the areas if the interval between two dates is more than 30 days. Shi et al. (1997) processed SIR-C data at L-band for the analysis of interferometric coherence and there by executed snow mapping. Limitations of this technique are that two images are to be acquired in a short interval with same viewing angle and swath mode. Presently, almost all satellites give data in the interval of 11 to 46 days.
%Coherence measurement using repeat pass SAR observations indicate low values for both wet and dry snow cover. The radar echoes get decorrelated for measurements between wet snow cover and bare ground. This happens mainly because radar signal can penetrate few centimeters in wet snow cover and hence the radar senses two different targets. In case of dry snow the dominant scattering is from the interface of snow ground. Decorrelation occurs in dry snow mainly due to volume scattering from the snow layer and changes the local incident angle. 
%The wet snow cover over the glacier and non-glacier area can be discriminated from other targets using degree of coherence because wet snow changes the scattering geometry and hence coherence is lost. But the decorrelation of time and baseline should be reasonable. Short time interval pair between snow and non-snow cover is better for separating snow from other targets (Singh et al. 2008b).
%Further cross polarization can improve classification, as the depolarized back scatterings are both sensitive to the surface roughness and show less angular dependence than the copolarised back scatterings. Investigations with polarimetric AIRSAR and SIR-C/X-SAR revealed improved capabilities for detecting wet snow and dry snow, in particular on fore-slopes (Rott et al. 1992, Shi et al. 1994, Rott 2001).
%Polarimetric SAR data provides much more information per pixel than single polarization data.  Here the radar transmits and receives energy in different polarizations like HHl, VV, HV, VH etc. An imaging radar polarimeter measures the amplitude and relative phase for every polarization state.  These complex measurements describe how the scattering mechanisms in each pixel transform the illuminating electromagnetic wave back to the receiving antenna.  The Polarization measurements are nearly independent of the estimate of illuminated area. Therefore, it is possible to map snow and glacier covered areas in Himalayan regions by using polarization measurements only, without topographic information.  The task is to select the measurements that are less sensitive to the variations in incident angle and best discriminate between snow, glacier ice and other surfaces. These types of measurements include:
%•	Specific polarization features, such as normalized cross product of scattering matrix elements and enhancement factors;
%•	Polarization impurity, coefficient of variation, fractional polarization and degree of polarization for a given incident polarization wave; and 
%•	Polarization ratios such as σ0hv/σ0vv and depolarization factors such as σ0hv/σ0hh and σ0hv/σ0vv where σ0 is the backscattering coefficient and subscripts vv, hh and hv are polarization states.
%Rott et al. (1992) used polarmetric data for snow cover mapping. Based on the depolarization properties of the radar signal, they used the following equation for mapping snow cover. If the backscattering power satisfies the following condition,
%PHV/PVV < m       snow covered.			   (2.4)
%where m = 0.04 was used. PHV is received power of cross-polarization and PVV is co-polarization data at C-band.  Shi et al. (1994) also used the same procedure and their results are in agreement with Rott et al. (1992).  
%Shi et al. (1994) have tested three measurements – the degree of the polarization for the incident wave in vertical polarization, the depolarization factor σ0hv/ σ0vv, and the normalized cross product of VV polarization – for the classification of surface features in glacier region of Otztal Alps in Austria, during an intense rainstorm in 1989 and during clear weather in 1991.  They have used AIR SAR polarimetric C band data. The degree of polarization is of the portion of the purely polarized power for a given polarization state of the incident wave, compared to the total scattered power.  
%•	The degree of polarization is 1.0 if the received power is purely polarized.  When multiple scattering is involved, the degree of polarization will be smaller. The measured degree of polarization from wet snow is close to 1.0 because multiple scattering is not important, but it is smaller in rock regions because the rough surface generates significant multiple scattering.
%•	The depolarization factor σ0hv/ σ0vv represents the effect of depolarized power and multiple scattering, but is directly proportional to multiple scattering.  Rock regions have larger values than wet snow because the rough surface generates multiple scattering interactions.
%•	Since the backscattering from a smooth surface and from snow is characterized by the maximum received power in VV polarization, the measurements of the normalized cross-product of the VV scattering matrix’s elements are large for wet snow.  However, the difference between the backscattered powers in VV and HH polarizations becomes smaller and finally identical as the surface becomes rougher.  The portion of the backscattered power of VV polarization in total backscattered power is much smaller for rock regions than for wet snow. 
%Discrimination is better in 1991 because the air snow interface was much smoother than during the rain storm in 1989 (Shi et al. 1994).The depolarization factor σ0hv/ σ0vv provides the best discrimination between all class pairs.  The degree of polarization of the vertically incident wave is a good discriminator between snow and glacier-covered areas and between snow and other surfaces (mainly rock, and moraine), but poorly separates glacier ice from other surfaces.  The normalized cross product of VV scattering matrix’s element is a good discriminator between snow or glacier ice with other targets, but does not discriminate well between them (Shi et al. 1994).
%The potential of multi-frequency, multi-polarisation SIR-C SAR data has been examined by Li and Shi (1995) to map snow covered Tienshen Mountain.  The results showed that C-band VV/HH images can discriminate between wet snow and snow free areas with 83-87% accuracy.  However some difficulty is encountered in discriminating glacier ice from snow and rock.  L band HH/VV images can separate snow-cover areas from other targets with 89-93% but there are problems to discriminate between rock and snow, and the cross polarization SAR images can be used to solve this problem.
%Shi and Dozier (1997) have demonstrated the capabilities of SIR-C/X-SAR to map seasonal snow cover in alpine region.  They have developed two types of classifiers based on classification trees.  The first type of the classifier was developed by using intensity measurements, polarization properties, and frequency ratios. It can map dry snow and discriminate dry from wet snow, but it requires topographic information for radiometric terrain correction and to reduce effects of local incident angle.  It is about 79% as accurate as a Landsat TM binary classification, but it underestimates total snow cover in regions of mixed pixels, especially forested regions.  Its performance on the two data takes where the snow was dry showed that only a few pixels were misclassified as wet snow.  The test on the wet snow data take showed that there was a little misclassification of the wet snow cover, but there were some problems in misclassification of short vegetation as dry snow at low elevations.  This type of misclassification error can be identified logically and corrected. The second type of classifier was developed based on polarization properties and backscattering ratios between different frequencies.  Since these measurements can be obtained correctly without radiometric terrain calibration, the classifier does not require topographic information and can be used to map wet snow. Its accuracy is 77% when compared with TM binary classification, but both underestimate total snow cover.
%Naglar and Rott (2005) show the capability of multi-temporal multiple polarizations for snow cover mapping. Single pass backscattering ratios of cross- versus co-polarized data of Envisat ASAR Alternating Polarisation Mode do not provide sufficient dynamic range for clear detection of snow. For multi-temporal segmentation, using snow and reference images of different dates, various combinations of polarization are suitable. Multi-temporal ratioing is not only restricted to images of the same polarization, but HH- and VV polarized data can be combined without any problems because of little polarisation differences in Alpine terrain. On the other hand, the combination of co- and cross-polarized backscattering is less suitable because the topographic effects are not fully compensated due to different angular dependence of backscattering.
%Ferro-Famil et al. (1999) suggested the methods based on airborne Radar polarimetry for studying changes in L- and C-band polarimetric descriptors (entropy, alpha angle and anisotropy) due to seasonal changes in snow covered area in Alps. Dry snow mapping has been carried out using multi frequency multi polarization SIR-C (L and C-band) temporal data (Martini 2005, Martini et al. 2005). Singh et al. (2008a) presented the capability of L-band full polarimetric ALOS-PALSAR data to discriminate snow from other targets using H/A/Alpha decomposition theorem and Wishat classifier and polarization fractional value. 
%2.4.2 Snow Wetness
%SAR backscattering coefficient is sensitive to many snow parameters that hydrologist use, especially free liquid water content in the snowpack because of the large dielectric contrast between ice and water in microwave spectrum (Shi et al. 1993). Several investigators (Arslan et al. 1999, 2005, Shi et al. 1993, Shi and Dozier 1995a & b, Shi and Dozier 2000, Strozzi and Matzler 1998, Singh et al. 2006, Singh et al. 2007, Venkataraman et al. 2007) reported that backscatter coefficient image of radar is extremely useful for the quantitative estimation of snow wetness (percentage of liquid water content in snow pack). 
%Suzuki et al. (1995) have carried out studies based on JERS-1 SAR data to understand the relation between the back scattering coefficient and various surface characteristics of snow pack in Hokkaido area, Japan.  L band, horizontal polarization SAR shows that the back scattering coefficient (σ0) decreases with increase in snow depth and snow wetness for a given snow density and surface roughness.  The variation in grain size does not seem to affect this relation.  In case of dry snow, there is very little change in σ0 with change in snow depth.  Using this relation it may be possible to derive snow wetness from backscatter coefficient if we know the snow depth.  However, it is not credible to derive a simple empirical relation between snow wetness and radar back scattering, as previous investigations have shown both negative and positive relationships between back scattering coefficients and snow wetness.  (Shi and Dozier 1995b).  This relationship is complicated, as surface roughness affects this relationship. Surface roughness affects the relationship between the backscattering coefficients and snow wetness.  For low wetness (≤ 3%) the dielectric contrast between air and snow is small and volume scattering dominates, so backscattering is now sensitive to surface roughness and decreases as wetness increases.  However, for wetter snow, backscattering becomes sensitive to surface roughness, because the surface-scattering component increases while the volume-scattering component decreases. Thus the relationship between back scattering and snow wetness is controlled by the scattering mechanism.  When the surface is smooth, volume scattering is the dominant scattering source.  As snow wetness increases, both the volume scattering albedo and the transmission coefficients greatly decrease.  This results in a negative correlation between the backscattering signals and snow wetness.  When the surface is rough, increasing snow wetness results in greatly increased surface scattering interaction and surface scattering becomes the dominant scattering process.  Therefore, a positive correlation between the backscattering signals and snow wetness will be observed (Shi et al. 1993). Moreover, the relationship between co polarization and snow wetness can be either positive or negative, depending on snow characteristics and surface roughness and on incidence angle. Singh et al. (2007) developed an empirical model for estimating snow wetness using HH polarized ASAR backscattering coefficients and measured wetness value. They found that ASAR backscattering coefficient is remaining approximately constant at snow wetness from 3.0% - 5.0 %.  This complexity of the relationship between the backscattering and snow wetness makes it unrealistic to develop a simple empirical relation, e.g., a regression equation, between the SAR signal and field measurements. 
%Shi et al. (1993) developed an inversion model for estimating snow wetness based on Small Perturbation Model (SPM). This model is applicable to 150-700 incident angle and large surface roughness. Further, Shi and Dozier (1995b) used fully polarimetric space shuttle borne SIR-C data to derive snow wetness by inverting a first order backscatter model (surface and volume scattering). By measuring three components (σ0vv= Backscattering Coefficient (BSC) of VV polarization, σ0hh = BSC of HH polarization and σ0vvhh=Re[VVHH*]) they were able to minimize effects from volume scattering and surface roughness (i.e. reduce the dimensionality of inversion problem) to estimate snow permittivity, which subsequently are related to snow wetness. Singh et al. (2006) also developed dielectric retrieval algorithm using first order surface scattering and volume scattering based on Physical Optics Model (POM). Further the dielectric constant has been related to snow wetness. Absolute error between Envisat-ASAR estimated and field measured value was observed to be 2.52% by volume. Kendra et al. (1998) conducted experiments to measure radar backscatter at C and X band of artificial snow of varying depths. The results of this study are amenable to comparison with predictions based on theoretical methods for modeling volume scattering media.  A direct polarimetric inversion approach has been described through which the characteristics of the snow medium are extracted from the measured data.  Kendra et al. (1998) have also collected backscatter values alongwith concurrent snow liquid water content (wetness) measurements in experimental snow packs. These measurements were used to confirm the validity of the algorithm developed by Shi and Dozier (1995b) for the retrieval of snow liquid water content. It has been found that the algorithm was able to precisely characterize the snow packs for their wetness. Niang et al. (2007) also developed statistical inversion model which performs retrieval of the snow wetness using ENVISAT- ASAR alternating polarization data.
%Shi and Dozier (1995b) model required fully polarimetric SAR data, but ENVISAT-ASAR have capability of data acquisition in dual polarization only. Hence, Singh and Venkataraman (2007) modified Shi and Dozier (1995b) snow wetness algorithm for dual polarization and the modified algorithm was implemented on ENVISAT-ASAR dual polarization data.  Model estimated wetness found to be quite in agreement with measured wetness with 2.8% absolute error. 
%2.4.3 Snow Density 
%Using Synthetic Aperture Radar (SAR) data, aforesaid algorithms (Shi et al. 1993, Shi and Dozier 1995, Singh et al. 2006, Singh and Venkataraman 2007, Niang et al. 2007, Singh and Venkataraman 2007) can be applied for only wet snow dielectric constant retrieval which can not be used for dry snow density estimation. When a snow pack is dry (at a temperature less than 0°C) microwaves easily penetrate the snow and the backscatter is largely a function of snow density. Singh et al. (2007) derived the empirical relationship between field measured snow density and ENVISAT-ASAR HH-polarization backscattering coefficient. They observed that this empirical relation becomes insensitive beyond snow density 0.35 g/cm3.    
%Interesting results have been obtained by Shi and Dozier (2000) using fully polarimetric L-band data from SIR-C/XSAR, obtaining snow density estimates. This algorithm maximizes the sensitivity to the incident angle and wave number while minimizing its sensitivity to the surface dielectric and roughness properties.  This algorithm does not require a priori knowledge of the subsurface dielectric and roughness properties.  It is insensitive to the SAR absolute calibration error but sensitive to the relative calibration error, especially at small incident angle (<30o).  At large incident angles (>50o) this algorithm becomes insensitive to the effect of the SAR relative calibration error.  This indicates that a better estimation accuracy of snow density can be obtained at larger incident angles than at small incident angles.  This model can be applied over a large range of incident angles (10o-70o).  The estimated snow density compared to field measurements shows an absolute RMSE of 42 kg m-3 and a relative 13% error.  This model can be applied to seasonal snow cover when the snow is dry.  Since the algorithm is based on the modification of the surface scattering problem, it can be applied where the subsurface is either rock or soil.  However it cannot be applied where the subsurface is dominated by volume scattering, as may be the case with snow-covered firn. Niang et al. (2007) also developed statistical inversion model which performs a simultaneous retrieval of the wetness and density using ENVISAT- ASAR alternating polarization data. 
