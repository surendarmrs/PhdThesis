\chapter{Summary and Conclusions}

In this thesis, the utilization of complete polarimetric SAR information for the estimation of snowpack parameters is described. Snow parameters in mountain areas are particularly sensitive to changes in environmental conditions. Timely gathering information about snow parameters and their temporal and spatial variability represents a significant contribution in climatology, local weather, avalanche forecasting and for the hydropower production in high mountainous areas. Conventional and ground-based methods represent only exact location measurements of field observations which may or may not be representative of a large area or basin. Due to the strong spatial and time-dependent dynamics of snow cover, frequent observation cycles are necessary. The sensitivity of microwave scattering to the characteristics of snowpack makes RADAR remote sensing a boon to understand a wide range of environmental issues related to the physical condition in high mountainous areas. Especially, the potential for retrieving snow parameters with a high spatial and/or temporal resolution corresponds to become an important input to snow avalanche forecasting, hydrological and meteorological modeling. 

Synthetic aperture radar (SAR) imaging technology is one of the most important advances in space-borne radar remote sensing during recent decades. In the present investigation, dual-~polarimetric (HH/VV) coherent TerraSAR-X (X-band) and full polarimetric Radarsat-2 (C-band) datasets have been used. Manali- Dhundhi region of Indian Himalaya is considered as a study area for this research work. Field data was collected synchronous with the satellite passes (Appendix-II). Snow parameters such as wetness, density, depth and snow permittivity have been measured using the snow fork instrument over the study area. Detailed analyses of  Microwave interaction with snow covered terrain and different scattering mechanisms are described in ~\cref{sec:2.3}, in order to understand the physical characteristics of snowpack parameters.

In this thesis, four major contributions were presented for the estimation of snow wetness, snow surface dielectric constant and snow density. The algorithms have been proposed and validated using polarimetric SAR data and near real time in-situ measurements. The methodologies have been clearly explained in ~\cref{sec:3} comprising of four separate sections. The results obtained from these approaches have been meticulously presented with detailed discussions in ~\cref{sec:4} pertaining to the corresponding sections.

\begin{itemize}
	\item As a first contribution, a new methodology for the snow wetness estimation from dual-~polarimetric (HH/VV) coherent high frequency (9.6 GHz) SAR data have been proposed. In this high frequency limit, the snow surface and the volume scattering components are considered to estimate the snow wetness. The surface wetness is derived from the simplified IEM model whereas the volume wetness is derived using the Rayleigh scattering assumption. The proposed methodology is applied to four TerraSAR-X datasets over the Indian Himalayan region. However, the validation has been carried out only for 23 Jan 2009 dataset, as the synchronous ground measurements were available only for this day. The overall mean absolute error for the proposed method is 1.65$\%$ by volume. It can also be seen that for 12, 18 and 24 January 2009 datasets the proposed method is able to correctly estimate the snow wetness for varying weather conditions. 
	
	\item A new model has been proposed to estimate snow wetness from full polarimetric SAR data as a second contribution. The proposed model was applied to Radarsat-2 fine resolution full-polarimetric data sets acquired over the Indian Himalayan region for three consecutive years. The results were comparable to the near real time in-situ measurements. The high accuracy of the proposed model is due to the utilization of the complete information from full polarimetric SAR data. The double unitary rotation of the full-polarimetric data has compensated the azimuth and the range slope effects which are predominant in mountainous regions. The effective wetness of the snowpack is estimated by using the normalized surface and the volume scattering powers derived from the G4U model based decomposition technique. This effective averaging of the surface and volume snow wetness using scattering powers is essential to account for different snow conditions. The comparison of the snow wetness derived from the proposed and the Shi-Dozier methods with the ground measurements indicated that the absolute error at 95$\%$ confidence interval were 1.3$\%$ and 2.6$\%$ by volume, respectively. The proposed method shows a better estimation of the snow wetness compared to the Shi-Dozier method.  
	
	\item A new methodology for snow surface dielectric constant estimation from  full-polarimetric C-band SAR data is proposed as a third contribution in this thesis. The dominant scattering type amplitude ($\alpha_{s1}$) is used to characterize dominant snow scattering mechanism along with the optimum degree of polarization ($m_{E}^{\mbox{\scriptsize opt}}$ and $p_{max}$) as a criteria for the selection of maximum surface scattering pixels which were used for the inversion of the snow surface dielectric constant. The AGU-Dop $m_{E}^{\mbox{\scriptsize opt}}$ have increased the number of pixels for inversion by approximately 9--10$\%$ compared to the original data while just $2\%$ more number of pixels were accounted for inversion using the Touzi optimum dop $p_{max}$. Moreover, these two Dop's can be suitably used hand-in-hand for advanced snowpack characterization analysis. The proposed algorithm was applied to three consecutive winter acquisitions of full-polarimetric fine resolution Radarsat-2 data over the Indian Himalayan region. The correlation coefficient between the measured and the estimated snow surface dielectric constant was found to be 0.95 at $95\%$ confidence interval with a root mean square error (RMSE) of 0.20.
	
	\item As a final contribution, a new methodology for snow density estimation from full-- polarimetric SAR data is proposed. The generalized volume parameter is derived from the double unitary transformation of the coherency matrix. This parameter was directly used to estimate the snowpack density. The proposed methodology is applied to three Radarsat-2 fine resolution full-polarimetric datasets acquired over the Indian Himalayan region for three consecutive years. Extensive field campaigns were conducted to collect near-real time snow density measurements along with the satellite data which have been used for validation of the proposed methodology. The mean absolute error (MAE) and the root mean square error (RMSE) are 0.027~gcm$^{-3}$ and 0.032~gcm$^{-3}$ respectively. The temporal snow density variation analysis, in the same season have been done  using the proposed method. 
	
	\item These research works have been assimilated in the HimSAR software toolbox which is under development and expansion for the cryospheric applications using polarimetric SAR data. This toolbox will be helpful for the cryospheric scientific community to utilize, explore and contribute to the further development of this open source toolbox.
	   
\end{itemize}
\section{Scope for future research}
\begin{itemize}
	\item The proposed snowpack parameters estimation algorithm can be extended for multi frequency SAR data by considering all possible scattering mechanisms.
	\item Particularly for the estimation of snow density where the snowpack volume scattering has only been considered for the Radarsat-2 C-band data. Furthermore, the algorithms should be modified for lower frequency data (S and L band) while considering the snow ground scattering component.
	\item In the proposed models vegetation impacts are not considered. these models will be further improved to consider snow pack under vegetation cover. 
	\item Other important snowpack parameters like snow depth and snow water equivalent can be estimated by utilizing the advanced polarimetric decomposition techniques. The available Bi-static TerraSAR-X/TanDEM-X full polarimetric SAR data also will be used for the snow depth estimation. 
	\item Furthermore, The generalized parameters derived from G4U decomposition can be used for other applications like Paleo channel mapping, soil moisture estimation, planetary surface exploration etc. This parameters can also be used for other cryospheric component studies. 
\end{itemize}
  




 

