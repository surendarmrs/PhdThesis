\chapter{Basics of Radar Polarimetry}
\section{Introduction}

Radar polarimetry is the merging of the technological concept of Radar (radio detection and ranging) with the fundamental property of the full vector nature of polarized (vector) electromagnetic waves imaging.(gulab thesis)

Spaceborne sensors provide valuable information about the earth's surface and environment. Active microwave sensors are of particular interest for this task due to their high resolution and their ability to image through clouds and at night. The conventional spaceborne imaging radars implemented for long-duration missions (the SAR sensors ERS-1 , ERS-2, JERS- 1 , Radarsat) operate in a single-frequency single polarization mode. The advances in technologies in the last two decades have led to the development of imaging radar polarimetry, where the complete, complex scattering matrix for every resolution element is measured. This capability enables the measurement of a target's polarization properties, thus permitting a much more detailed understanding of the electromagnetic scattering process. Radar backscatter is strongly influenced by objects comparable in size to the radar wavelength. Therefore, a polarimetric sensor operating in various frequency bands provides information about the imaged target over a wide range of scales.


The field of synthetic aperture radar changed dramatically in the early 1980s with the introduction of advance radar techniques, such as polarimetry and interferometry. While both of these techniques had been demonstrated much earlier, radar polarimetry only became an operational research tool with the introduction of the NASA/JPL Airborne Synthetic Aperture Radar (AIRSAR) system in the early 1980s. Radar polarimetry was proven from space with the two Spaceborne Imaging Radar C-band and X-band (SIR-C/X) SAR flights on board the space shuttle Endeavour in April and October 1994. In this chapter, we describe the basic principles of SAR polarimetry and, thereby, provide tools necessary to understand SAR polarimetry applications, such as land classification.

