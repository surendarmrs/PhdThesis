
Snow in mountain areas is especially sensitive to changes in environmental conditions due to their proximity to melting conditions which may lead to hazards like avalanche, floods etc.  Timely information about snow parameters and their temporal and spatial variability is an important factor in climatology, local weather studies, avalanche forecasting and for the hydropower production. In this thesis, polarimetric Synthetic Aperture Radar (SAR) data is utilized for the estimation of snowpack parameters. SAR imagery technology is one of the most important advances in space-borne radar remote sensing during recent decades. In the present investigations, dual-~polarimetric (HH/VV) coherent TerraSAR-X (X-band) and full polarimetric Radarsat-2 (C-band) datasets are used.

In this thesis, four new methods are presented for the estimation of snow wetness (from dual-coherent and full polarimetric SAR data), snow surface dielectric constant and snow density. These proposed algorithms are validated using near real time in-situ measurements. The Manali-Dhundhi area of the Indian Himalayan region is considered as a study area for this research work. 

Snow wetness estimation from dual-polarimetric coherent (HH/VV) SAR data is proposed as a first study. Surface and volume are the dominant scattering components in the wet-snow conditions. These components, with a limit of penetration depth for high frequency SAR are taken into consideration to estimate the snowpack wetness. In this new methodology, the snow surface wetness is estimated using the simplified IEM scattering model and the snow volume wetness is estimated under the Rayleigh scattering assumption. The estimated snow wetness is validated using the in-situ measurements, which is collected in synchronous with the satellite pass. In this study, the dual-polarimetric coherent TerraSAR-X data acquired over Solang, Himachal Pradesh, India on 23 January 2009 is used for the validation. The snow wetness estimated by the proposed method shows the mean absolute error is 1.65$\%$ by volume. Along with this dataset, three more TerraSAR-X datasets acquired on 12, 18 and 24 January 2009 have been used to infer the snow wetness changes over the study area. 

A novel snow wetness estimation algorithm is developed for full polarimetric SAR data in the subsequent study. The generalized four component polarimetric decomposition with unitary transformation (G4U) have been utilized. The generalized surface and volume parameters derived from G4U are used to invert snow surface and volume dielectric constants using the Bragg coefficients and Fresnel transmission coefficients respectively. The snow surface and volume wetness are then estimated using a standard empirical relationship. The effective snow wetness is derived from the weighted averaged surface and volume snow wetness. The weights are derived from the normalized surface and volume scattering powers obtained from the generalized  full-polarimetric SAR decomposition method. The Radarsat-2 fine resolution full-polarimetric datasets along with the near-real time in-situ measurements were used to validate the proposed method. The snow wetness derived from the SAR data by the proposed methodology with in-situ measurements indicated that the absolute error at 95$\%$ confidence interval is 1.3$\%$ by volume.

A new method for the estimation of snow surface dielectric constant from polarimetric SAR data is presented in this thesis. The dominant scattering type magnitude proposed by Touzi et. al., is used to characterize scattering mechanism over the snowpack; whereas two methods have been used to obtain the optimized degree of polarization for a partially polarized wave: (1). the Touzi optimum degree of polarization given in Touzi et. al., 1992. In this approach, the maximum $(p_{max})$ and the minimum $(p_{min})$ degree of polarizations are obtained along with the optimum transmitted polarizations $(\chi_{t}^{\mbox{\scriptsize opt}},\psi_{t}^{\mbox{\scriptsize opt}})$. (2). the Adaptive Generalized Unitary (AGU) transformation based optimum degree of polarization ($m_{E}^{\mbox{\scriptsize opt}}$) proposed in Bhattacharya et. al., 2015. This optimum degree of polarization is obtained either by a real or a complex unitary transformation of the 3$\times$3 coherency matrix. These two degree of polarizations are used and compared in this study as a criteria to select the maximum number of pixels with surface dominant scattering. These pixels were then used to invert the snow surface dielectric constant. The $m_{E}^{\mbox{\scriptsize opt}}$ have increased the number of pixels for inversion by $\approx 9-10\%$ compared to the original data. On the other hand, it was observed that the Touzi maximum degree of polarization ($p_{max}$) have increased the number of pixels for inversion by $\approx 2\%$ compared to that of $m_{E}^{\mbox{\scriptsize opt}}$. The proposed methodology is applied to Radarsat-2 PolSAR C-band datasets over the study region. It is observed that the correlation coefficient between the measured and the estimated snow surface dielectric constant is 0.95 at 95$\%$ confidence interval with a root mean square error (RMSE) of 0.20. 

Along with the above snow parameter estimation algorithms, a snow density estimation algorithm is also proposed for C-band full-polarimetric SAR data. The generalized four component polarimetric decomposition with unitary transformation (G4U) based generalized volume parameter is utilized to invert snowpack dielectric constant using the Fresnel transmission coefficients. The snow density is then estimated using a standard empirical relationship. The near-real time in-situ measurements were collected along with the Radarsat-2 fine resolution full-polarimetric SAR data to validate the proposed method. The mean absolute error (MAE) of the proposed method is 0.027~gcm$^{-3}$ and the root mean square error (RMSE) is 0.032~gcm$^{-3}$. The snow density variations within a season were also analyzed using multi-temporal Radarsat-2 data. 

Altogether, the research work has culminated in the development of HimSAR, a standalone snowpack parameters estimation toolbox. This toolbox aims to facilitate the accessibility of multi-frequency polarimetric SAR data for cryospheric applications. This will be helpful for the scientific community to utilize, explore and contribute to the further development of this open source toolbox. Furthermore, the toolbox also caters to reproducible research among the cryospheric community.    